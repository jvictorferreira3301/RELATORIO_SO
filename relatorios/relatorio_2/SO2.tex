\documentclass[
	% -- opções da classe memoir --
	12pt,				% tamanho da fonte
	openright,			% capítulos começam em pág ímpar (insere página vazia caso preciso)
	oneside,			% para impressão em frente e verso. Oposto a oneside
	a4paper,			% tamanho do papel.
	% -- opções da classe abntex2 --
	chapter=TITLE,		% títulos de capítulos convertidos em letras maiúsculas
	%section=TITLE,		% títulos de seções convertidos em letras maiúsculas
	%subsection=TITLE,	% títulos de subseções convertidos em letras maiúsculas
	%subsubsection=TITLE,% títulos de subsubseções convertidos em letras maiúsculas
	% -- opções do pacote babel --
	english,			% idioma adicional para hifenização
	french,				% idioma adicional para hifenização
	spanish,			% idioma adicional para hifenização
	brazil				% o último idioma é o principal do documento
	]{abntex2}
% ---
% Pacotes básicos 
% ---
\usepackage{lmodern}			% Usa a fonte Latin Modern
\usepackage{mathptmx}			% Usa a fonte Times New Roman
\usepackage[T1]{fontenc}		% Selecao de codigos de fonte.
\usepackage[utf8]{inputenc}		% Codificacao do documento (conversão automática dos acentos)
\usepackage{lastpage}			% Usado pela Ficha catalográfica
\usepackage{indentfirst}		% Indenta o primeiro parágrafo de cada seção.
\usepackage{color}				% Controle das cores
\usepackage{graphicx}			% Inclusão de gráficos
\usepackage{subcaption}				% Inclusão de gráficos lado a lado
\usepackage{microtype} 			% para melhorias de justificação
\usepackage{tabularx,ragged2e}	% Para inserir tabelas
\usepackage{multirow}			% Para mesclar células
\usepackage[dvipsnames,table,xcdraw]{xcolor}		% Permite adicionar cores nas linhas de tabelas
\usepackage{fancyvrb}			% Permite adicionar arquivos de texto
\usepackage[portuguese, ruled, linesnumbered]{algorithm2e} % Uso de algoritmos
\usepackage{amsfonts}			% Permite usar notação de conjuntos
\usepackage{amsmath}			% Permite citar equações
\usepackage{amsthm}				% Permite criar teoremas e experimentos
\usepackage[font={bf, small}, labelsep=endash, labelfont=bf]{caption}	% Faz legenda de figuras ficarem em negrito
\usepackage{cancel}				% Permite fazer expressão tendendo a zero
\usepackage{epstopdf}			% Converte eps para pdf
\usepackage[final]{pdfpages}
\usepackage{hyperref}
\usepackage{fancybox}


\newcolumntype{L}{>{\RaggedRight\arraybackslash}X}
% ---
% ---
% Pacotes adicionais, usados apenas no âmbito do Modelo Canônico do abnteX2
% ---
\usepackage{lipsum}				% para geração de dummy text
% ---
% ---
% Pacotes de citações
% ---
%\usepackage[brazilian,hyperpageref]{backref}	 % Paginas com as citações na bibl
\usepackage[alf, abnt-emphasize=bf]{abntex2cite}	% Citações padrão ABNT
% ---
% Customizações para o layout da UFPA
% ---
\usepackage{modelo-ufpa/ufpa}
% Muda o título de lista de ilustrações para lista de figuras
\addto\captionsbrazil{%
  \renewcommand{\listfigurename}%
    {Lista de Ilustrações}%
	\renewcommand{\listtablename}%
    {Lista de Tabelas}%
}
% Permite utilizar figuras sem precisar colocar o caminho absoluto
\graphicspath{{imagens/}}
% Define o ambiente de experimentos
\theoremstyle{definition}
\newtheorem{experimento}{Experimento}[section]
\newcommand{\experimentoautorefname}{Experimento}


% --------------------------------------------------------------
% Informações do TRABALHO
% --------------------------------------------------------------
\universidade{UNIVERSIDADE FEDERAL DO PARÁ}
\instituto{INSTITUTO DE TECNOLOGIA}
\faculdade{FACULDADE DE COMPUTAÇÃO E TELECOMUNICAÇÕES}
%\curso{CURSO DE BACHARELADO EM SISTEMAS DE INFORMAÇÃO}
\titulo{RELATÓRIO DE SO}
\autor{
%\begin{tabular}{l l}
    DAVID PINHEIRO DE SOUSA - 202207040045 \\
    JOAO VICTOR SANTOS BRITO FERREIRA - 202207040028 \\
    JOEL TAVARES MIRANDA - 202206840054 \\
    KAUAN MIRANDA TAVARES - 202206840033 \\
    MARCO ANTONIO DO ESPIRITO SANTO MAUES JUNIOR - 202206840038 \\
%\end{tabular}
}
\local{Belém}
\data{2023}
\orientador{Prof. Dr. Diego Lisboa Cardoso}
\tipotrabalho{Monografia}

% o nome da instituição e a área de concentração 
\preambulo{Relatório do trabalho 2 de Sistemas Operacionais.}
%\sobrenome{Sobrenome}
%\nome{Nome}
%\palavraschave
%\datadadefesa{Data da Defesa: 09 de Março de 2017}%07 de Dezembro de 2016}
\conceito{Conceito: Excelente}
\faculdadedoorientador{Faculdade de Computação e Telecomunicações - UFPA}
%\primeiromembrodabanca{Prof. Dr. Nome Sobrenome}
%\faculdadedoprimeiromembrodabanca{}
%\segundomembrodabanca{}
%\faculdadedosegundomembrodabanca{}
% -------------------------------------------------------------------------
% ---
% Configurações de aparência do PDF final
% alterando o aspecto da cor azul
\definecolor{blue}{RGB}{41,5,195}
% informações do PDF
\makeatletter
\hypersetup{
     	%pagebackref=true,
		pdftitle={\imprimirtitulo}, 
		pdfauthor={\imprimirautor},
    	pdfsubject={\imprimirpreambulo},
	    pdfcreator={LaTeX with abnTeX2},
		pdfkeywords={\imprimirpalavraschave}, 
		colorlinks=true,       		% false: boxed links; true: colored links
    	linkcolor=black,          	% color of internal links
    	citecolor=black,        		% color of links to bibliography
    	filecolor=magenta,      		% color of file links
		urlcolor=blue,
		bookmarksdepth=4,
        breaklinks=true
}
\makeatother
% --- 
% Espaçamentos entre linhas e parágrafos 
% --- 
% O tamanho do parágrafo é dado por:
\setlength{\parindent}{1.3cm}
% Controle do espaçamento entre um parágrafo e outro:
\setlength{\parskip}{0.2cm}  % tente também \onelineskip
% compila o indice
% ---
\makeindex
% ---

% -------------------------------------------------------------------------
% ---------------------------INICIO DO DOCUMENTO---------------------------
% -------------------------------------------------------------------------
\begin{document}
% Seleciona o idioma do documento (conforme pacotes do babel)
\selectlanguage{brazil}
% Retira espaço extra obsoleto entre as frases.
\frenchspacing 
% ----------------------------------------------------------
% ELEMENTOS PRÉ-TEXTUAIS
% ----------------------------------------------------------
% \pretextual

% ---
% Capa
% ---
\imprimircapa
% ---

% ---
% Folha de rosto

\imprimirfolhaderosto

\newpage

\setlength{\absparsep}{18pt} % ajusta o espaçamento dos parágrafos do resumo

\pdfbookmark[0]{\contentsname}{toc}
\tableofcontents*
\cleardoublepage
% ---
% ---------------------------------------------------------
% ELEMENTOS TEXTUAIS
% ----------------------------------------------------------
\textual

% ----------------------------------------------------------
% Introdução
% ----------------------------------------------------------

\chapter{Introdução}

Este trabalho aborda a implementação de concorrência e 
sincronismo por meio da criação de processos e threads. O 
trabalho consiste em uma série de tarefas que exploram diferentes 
aspectos da criação de processos, comunicação entre processos e o 
uso de threads em um ambiente Linux. Através dessas atividades, 
pretendemos aprofundar nossa compreensão de como a concorrência é 
gerenciada em sistemas operacionais e como diferentes abordagens afetam 
o desempenho e o comportamento do programa.


\chapter{Objetivos}

O principal propósito deste trabalho é:

\begin{itemize}
    \item \textbf{Criação de Processos}:
    \begin{itemize}
        \item Implementar a criação de processos de acordo com as hierarquias e requisitos especificados, incluindo cenários com avô, pai e filho, bem como múltiplos filhos.
    \end{itemize}

    \item \textbf{Comunicação entre Processos e Threads}:
    \begin{itemize}
        \item Explorar a comunicação entre processos e threads, incluindo o compartilhamento de variáveis e recursos entre eles.
    \end{itemize}

    \item \textbf{Comparação de Desempenho}:
    \begin{itemize}
        \item Comparar o desempenho da criação de processos e threads para realizar as tarefas do programa. Calcular a média de tempo de 20 simulações para obter uma análise precisa.
    \end{itemize}

    \item \textbf{Teste de Limites Máximos}:
    \begin{itemize}
        \item Testar e identificar os limites máximos da quantidade de processos e threads que podem ser disparados em um ambiente Linux. Explorar as configurações relacionadas.
    \end{itemize}
\end{itemize}

  

\chapter{Desenvolvimento (Programa explicado e Cópias das telas do emulador.)}



\section{}



\subsection{Código}




\chapter{Conclusão}
\label{conclusao}





\postextual

\bibliography{bibliografia}

\cite{tanenbaum2010sistemas}

\end{document}
